\documentclass[a4paper,10pt,titlepage,oneside,openright]{book}
	 \textheight = 24cm
	 \textwidth = 18cm
	 \topmargin = -1cm
	 \oddsidemargin = -1cm
	 
%\input{../../../header-apuntes.tex} %Ponemos ruta de la cabecera
\setcounter{secnumdepth}{3} % Para enumerar hasta subsubsection
\setcounter{tocdepth}{3} % Para enumerar hasta subsubsection 
 
\usepackage{indentfirst} % Paquete para iniciar el primer párrafo de una sección con sangría
\usepackage[utf8]{inputenc}
\usepackage[none]{hyphenat} %Paquete para indicar que no separe las palabras
\usepackage{graphicx} %Paquete para incluir imágenes
\usepackage{fancyhdr} %Paquete para modificar los pies de página y encabezados
\usepackage{listings} %Paquete para poner código de programación
\usepackage{color} %Paquete para definir los colores
\usepackage[hidelinks]{hyperref} %Paquete para que el índice tenga referencia
\usepackage{wrapfig} %Paquete para poner las imágenes con el texto en un lado.
\usepackage{multicol} %Paquete para poner texto en dos columnas
\usepackage{enumitem} %Paquete para poner en enumerate: 1.2, 1.3,...
\usepackage[nottoc]{tocbibind} %Paquete para incluir la bibliografía en el índice
\usepackage{colortbl} %Paquete para poner colores en las celdas
\usepackage{tocloft} %Paquete para personalizar el índice
\usepackage{textcomp} %Paquete para insertar símbolos raros
\usepackage{amssymb} %Paquete para insertar flechas raras
\usepackage{enumitem} %Paquete pare personalizar los enumerate
%\usepackage{kbordermatrix} %Parquete para poner bordes en las matrices

\definecolor{blue-violet}{rgb}{0.54,0.17,0.89}
\definecolor{darkcerulean}{rgb}{0.03, 0.27, 0.49}
\definecolor{ceruleanblue}{rgb}{0.16, 0.32, 0.75}
\definecolor{dkgreen}{rgb}{0,0.6,0}
\definecolor{gray}{rgb}{0.5,0.5,0.5}
\definecolor{mauve}{rgb}{0.58,0,0.82}
\definecolor{gray97}{gray}{.97}
\definecolor{gray75}{gray}{.75}
\definecolor{gray45}{gray}{.45}

\lstset{frame=Ltb,
  language=SQL,
  framerule=0pt,
  aboveskip=0.5cm,
  framextopmargin=3pt,
  framexbottommargin=3pt,
  framexleftmargin=0.4cm,
  framesep=0pt,
  rulesep=.4pt,
  backgroundcolor=\color{gray97},
  rulesepcolor=\color{black},
  %
  stringstyle=\color{mauve},
  showstringspaces = false,
  basicstyle=\scriptsize\ttfamily,
  commentstyle=\color{gray45},
  keywordstyle=\color{darkcerulean},
  %
  numbers=left,
  numbersep=15pt,
  numberstyle=\tiny,
  numberfirstline = false,
  breaklines=true,
  escapeinside=||,
  %
  literate={«}{{\guillemotleft}}1
           {»}{{\guillemotright}}1
           {é}{{\'e}}1
           {í}{{\'i}}1
           {ó}{{\'o}}1
           {ú}{{\'u}}1
           {á}{{\'a}}1
           {ñ}{{\~n}}1
           {Ñ}{{\~N}}1
           {¿}{{?`}}1
}

% minimizar fragmentado de listados
\lstnewenvironment{listing}[1][]
{\lstset{#1}\pagebreak[0]}{\pagebreak[0]}

\renewcommand{\contentsname}{Índice}
\renewcommand{\partname}{TEMA}
\renewcommand{\chaptername}{Tarea}
\renewcommand{\thesection}{\arabic{section}}
\renewcommand{\listtablename}{Índice de tablas}
\renewcommand{\tablename}{Tabla}
\renewcommand{\figurename}{Figura}
\renewcommand{\bibname}{Bibliografía}
% \renewcommand{\cftsecfont}{\bfseries} % Poner las secciones en negrita en el índice

% Personalización del itemize
\renewcommand{\labelitemi}{\textendash}
\renewcommand{\labelitemii}{\textperiodcentered}
\renewcommand{\labelitemiii}{\textasciicircum}

	
\graphicspath{ {images/} } %Indicamos la carpeta donde están las imágenes

\newcounter{ejemplo} % Contador para los ejemplos
%\addtocounter{ejemplo}{1} % Sumamos 1

\newcommand{\master}{M. Big Data \& Business Analytics}
\newcommand{\module}{Módulo V: Inteligencia de Negocio y Visualización}


\begin{document}
%--------------------------------------------------------------------
\thispagestyle{empty}
\begin{figure}[h]
\includegraphics[scale=0.3]{logo-imf.png} \hspace{80mm}
\includegraphics[scale=0.25]{logo-ucjc.png}
\centering
\end{figure}

\vspace{5mm}

\begin{center}
\rule{150mm}{0.1mm} \\
\vspace{5mm}
\begin{Huge}
 \textsc{Máster en \\ ~ \\ Big Data \& Business Analytics}
\end{Huge}
\vspace{5mm} \\
\rule{150mm}{0.5mm}

\vspace{20mm}

\begin{huge}
  \textsc{Módulo V: \\ Inteligencia de Negocio y \\ Visualización} \\ \vspace{15mm}
\end{huge}

\begin{LARGE}
 Documentación del caso práctico\\ \vspace{5mm}
 {\large 31 de Octubre del 2018}
\end{LARGE}

\vspace{25mm}

\begin{Large}
\begin{center}
\begin{tabular}{ccc}
\textbf{Autor} \\
José María Sánchez Salas \\
\textit{josemaria.sanchezsalas@gmail.com}
\end{tabular}
\end{center}
\end{Large}
\end{center}
%--------------------------------------------------------------------
\newpage
\thispagestyle{empty}
\section*{}
\newpage

\thispagestyle{empty}

\tableofcontents
\addtocontents{toc}{\textbf{\module}}

\newpage
\thispagestyle{empty}
\section*{}
\newpage

\pagestyle{fancy}
\fancyhf{}
\fancyhead[LE,RO]{\module}
\fancyhead[RE,LO]{\master}
\fancyfoot[CE,CO]{\thepage}

% \lhead[]{} - \chead[]{} - \rhead[]{}
\renewcommand{\headrulewidth}{0.5pt} % --> Definimos el grosor de la línea
\renewcommand{\footrulewidth}{0.5pt}

%--------------------------------------------------------------------
\medskip
\section{Introducción}
Este documento contiene la explicación de mi solución propuesta al caso práctico del {\module}, del {\master} impartido por IMF Business School. El documento contiene tres secciones principales: en la primera se muestra el enunciado el caso práctico, en la segunda se muestra la solución propuesta a todas las preguntas planteadas y por último, la tercera sección muestra las conclusiones a las que se ha llegado en la realización de este caso práctico.



%--------------------------------------------------------------------
\medskip
\section{Enunciado}
El departamento antifraude de una compañía de Mystery Shopping desea hacer un seguimiento y
analizar la información relativa a las encuestas que realiza en los distintos centros de sus clientes. Para
ello, el cliente solicita:

\begin{itemize}
    \item Un análisis y diseño del Data Warehouse que daría respuesta a los usuarios analíticos del
departamento antifraude, suponiendo que los usuarios aún no tienen claro el tipo de análisis
que quieren realizar.
    \item Partiendo del análisis y diseño previo realizado y usando Pentaho Data Integration, se debe
realizar la implementación del proceso ETL con el objetivo de:
    \begin{itemize}
        \item Identificar y extraer los datos de las fuentes.
        \item Procesar los datos y aplicar procesos de limpieza y calidad del dato.
        \item Generar y cargar los datos en el modelo físico de estrella identificado en la fase de diseño.
    \end{itemize}

    \item Posteriormente, partiendo del análisis y diseño previo realizado y conociendo ya la tecnología
seleccionada, en este caso Pentaho Business Analytics, ha de realizarse una implementación
ágil del modelo multidimensional.
\end{itemize}


El objetivo en este caso es la implementación del modelo multidimensional sobre diseño del Data
Warehouse que daría respuesta a los usuarios analíticos del departamento antifraude, suponiendo que los
usuarios aún no tienen claro el tipo de análisis que quieren realizar.




%--------------------------------------------------------------------
\medskip
\section{Solución}



%--------------------------------------------------------------------
\medskip
\subsection{Análisis de fuentes}

%--------------------------------------------------------------------
\medskip
\subsection{Análisis funcional y diagrama de arquitectura de flujo de datos}

%--------------------------------------------------------------------
\medskip
\subsection{¿Qué arquitectura de referencia usaría? Justifique la respuesta}
Dada las características del análisis propuesto en la sección \ref{02}, la arquitectura que se va a emplear es la de datos en tres niveles:
\begin{itemize}
 \item \textbf{Datos}. En este nivel se encontrarían el Origen de datos y el Data Warehouse de la Figura \ref{02-image}.
 \item \textbf{Aplicación}. En este nivel se encontrarían todas las funcionalidades de las herramientas OLAP.
 \item \textbf{Presentación}. Y en el último nivel se encontraría toda la parte de la Visualización de la Figura \ref{02-image}.
\end{itemize}

%--------------------------------------------------------------------
\medskip
\subsection{¿Qué tecnología OLAP usaría? Justifique la respuesta}
Dadas las características de la fuente de datos, expuestas en la sección \ref{01}, la tecnología OLAP que se va a usar va a ser ROLAP. Además de las características propias de la fuente de datos, se ha elegido ROLAP porque los usuarios no saben el tipo de análisis que van a realizar, por lo que al usar ROLAP se gana flexibilidad a la hora de la generación de análisis, además de que también permite poder realizar cualquier consulta ad-hoc sobre cualquiera de los atributos que contiene los datos.

%--------------------------------------------------------------------
\medskip
\subsection{Si se utiliza ROLAP, ¿cuál de estos dos modelos se ajustaría mejor: modelo de estrella o el de copo de nieve?}

%--------------------------------------------------------------------
\medskip
\subsection{Si se utiliza ROLAP, hay que identificar y justificar si existe algún proceso de desnormalización de información que se deba realizar}

%--------------------------------------------------------------------
\medskip
\subsection{Si se utiliza ROLAP, se debe incluir un diseño conceptual a modo explicativo junto con un diagrama}

%--------------------------------------------------------------------
\medskip
\subsection{Si se utiliza ROLAP, se debe incluir un diseño modelo lógico}

\newpage
%--------------------------------------------------------------------
\medskip
\subsection{Si se utiliza ROLAP, se debe incluir un diseño modelo físico}

\newpage
%--------------------------------------------------------------------
\medskip
\subsection{Realizar la implementación del proceso ETL para generar y poblar el modelo multidimensional diseñado en los apartados anteriores}

%--------------------------------------------------------------------
\medskip
\subsection{Implementación de modelo multidimensional diseñado en los puntos anteriores}
\label{11}
Para realizar la implementación del modelo multidimensional se ha hecho uso de la herramienta Wizard proporcionada y facilitada en el módulo. Con ella, y basándonos en el modelo lógico descrito en la sección \ref{08}, se ha implementado la implementación del modelo multidimensional que se muestra en la Figura \ref{modelo-multidimensional}.

\begin{figure}[!th]
\includegraphics[scale=0.5]{modelo-multidimensional-1.png}
\includegraphics[scale=0.5]{modelo-multidimensional-2.png}
\centering
\caption{Implementación del modelo multidimensional utilizando Wizard.}
\label{modelo-multidimensional}
\end{figure}

%--------------------------------------------------------------------
\medskip
\subsection{Análisis de modelo}





%--------------------------------------------------------------------
\medskip
\section{Conclusiones}
Las conclusiones que se pueden obtener de este caso práctico es que se aleja, en mi opinión, bastante de lo enseñado en el módulo en ciertos puntos. Sin embargo, sí se acerca a lo que suele pedirse en el mundo real y eso es muy importante, ya que a los alumnos nos saca de nuestra zona de comfort estudiantil. Además, ayuda a comprender e interiorizar mejor los conceptos teóricos explicados en el módulo.\\

Personalmente, considero que este caso práctico ha sido uno de los que más me ha costado realizar, debido a que no estoy tan familiarizado este tema como puede ser la programación, de la que ha ido más de la mano el resto de módulos. He aprendido mucho realizando el caso práctico, y todo ha sido gracias al tutor, Jesús, que ha sabido ayudarme y guiarme desde el principio, cuando no sabía ni por dónde empezar.
















\end{document}
