
%------------------------------------------------   
\section{Forecasting}





%------------------------------------------------   
\subsection{Preprocesamiento}

Antes de poder aplicar el algoritmo de forecasting ARIMA \citep{arima}, el dataset debe ser preprocesado, al igual que se realizó con el clustering en la sección \ref{sec:clustering-preprocessing}. Sin embargo, no se ha realizado el mismo preprocesamiento. A continuación se detallan los pasos realizados en este preprocesamiento:

\begin{itemize}
 \item \textbf{Rellenar valores nulos}. En el dataset se han encontrado tres tipos de valores nulos: 
 \begin{itemize}
  \item \textbf{Valores de formato fecha}. Han sido rellenados con el valor \code{01/01/1990}.
  \item \textbf{Valores de formato texto}. Han sido rellenados con el valor de la cadena vacía.
  \item \textbf{Valores de formato texto asociados a identificadores}. Han sido rellenados con el valor \code{-1}.
 \end{itemize}
 
 \item \textbf{Eliminar de los registros} con valor distinto de \code{01/01/1990} en los campos \\ \code{DiscontinuedDate} y \code{ChemicalDateRemoved}, pues solo se precisan de aquellos registros de cosméticos que tengan productos químicos y no hayan sido retirados del mercado.

 \item \textbf{Agrupar por el campo} \code{InitialDateReported} sumando los valores del campo \\ \code{ChemicalCount}. Ya que el objetivo de la aplicación de forecasting es poder obtener una predicción de la cantidad de productos químicos que serán reportados en el futuro.
\end{itemize}

Tras aplicar este preprocesamiento nos queda un dataset con 1.863 registros y 2 características: \code{InitialDateReported} y \code{ChemicalCount}.




%------------------------------------------------   
\subsection{Obtención de los datasets de entrenamiento y validación}


Para poder realizar el forecasting, se necesita tener datos de entrenamiento y datos de validación (en adelante, \code{dataset} y \code{validation}, respectivamente). La obtención de estos datasets está ligada a que el dataset \citep{dataset} es incremental y se tienen almacenados dos versiones del dataset, como se ha comentado en la sección \ref{sec:data-downloading}. \\ 

Así, se disponen de las siguientes versiones del dataset:

\begin{itemize}
 \item Carpeta \code{src/data/} \citep{master}, donde se almacena la versión más reciente del dataset.
 \item Carpeta \code{src/data\_backup/} \citep{master}, donde se almacena la versión anterior del dataset.
\end{itemize}












\newpage
%------------------------------------------------   
\subsection{Aplicación del algoritmo ARIMA}
















