

La preocupación del ser humano por su aspecto y su cuidado es algo que se ha ido manteniendo a lo largo de los siglos. Precisamente, en el Antiguo Egipto se encuentran los primeros vestigios de la elaboración y utilización de diferentes productos cosméticos, utilizando para ello productos naturales, como plantas aromáticas. \\

Hoy en día, los cosméticos han vuelto a incorporar productos químicos dentro de sus fórmulas y se utilizan miles de estos compuestos a los que se le atribuyen multitud de propiedades. \\

En este Trabajo Fin de Máster (TFM) se expone el análisis de los productos químicos en los cosméticos. Los objetivos subyacentes a este análisis son: primero, encontrar una clasificación de los productos químicos presentes en los cosméticos; segundo, encontrar qué productos químicos son los más frecuentes en los cosméticos y qué cosméticos presentan mayor número de productos químicos; y tercero, obtener una predicción de la cantidad de productos químicos dañinos que contendrán los futuros cosméticos. \\

Para la realización del primer objetivo, se ha aplicado el algoritmo K-Means como técnica de clustering; para la consecución del segundo objetivo, se han aplicado diversas técnicas de data science como el análisis exploratorio de datos; y para la consecución del tercer objetivo, se ha aplicado el algoritmo ARIMA como técnica de forecasting.


